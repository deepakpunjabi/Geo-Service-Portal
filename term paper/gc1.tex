\documentclass[12pt]{IEEEtran}
\usepackage{graphicx}   % Written by David Carlisle and Sebastian Rahtz
\usepackage{fixltx2e}
\usepackage{url}        % Written by Donald Arseneau
\usepackage{amsmath}    % From the American Mathematical Society

% Your document starts here!
\begin{document}
\markboth{Geo-Crawler}{}

% Write abstract here
\begin{abstract}
In recent times, the increase in spatial data and services is vastly increased. To deal with this huge amount of data some kind of indexing is required. But this data is heterogeneous in nature and there are other problems as the scale of the web. We try to deal with these problems and try to implement a crawler based on WFS standards of OGC web services. Performance evaluation is an important aspect to judge the semantics used for the system.
\end{abstract}
\linespread{1.6}

% Each section begins with a \section{title} command
\section{Introduction}
	% \PARstart{}{} creates a tall first letter for this first paragraph
	\IEEEPARstart{T}{he} attribute based access control mechanism is an emerging technique to implement authorization model for resources in an organization. As the name may indicate, ABAC makes use of the attributes of the object and that of requestor. In this document various components of ABAC are described.
    \subsection{Document Structure}
    The document is divided into 4 parts, each emphasizing various aspects of ABAC.\\
    Part 1 consists of the terminologies that are used in the document to understand ABAC concepts better.\\
    Part 2 consists of Basic introduction of ABAC.\\
    Part 3 consists of implementation of ABAC in a large enterprise.\\
    Part 4 consists of various aspects that need to be considered as various levels of implementation of ABAC in an enterprise.
    
	%\PARstart{T}{his} section introduces the topic and leads the reader on to the main part.

% introduction to spatial things
\section{Spatial Data}
The term Geo comes from geography. Geography stores all the information of location and shape of the object in the spatial data. Spatial data stores the relationships between these data. It can also be easily mapped to a map. Geo-server provides various kinds of functionality to this type of data.

\subsection{Classification of Spatial Data}
Different types of object under the class geometry are as below. All the major Geo-spatial service providers and vendors provide this kind classification.The primitive four data types in spatial data are point, curve, surface and GeomCollection.

\begin{itemize}
\item \textbf{Point}\\
Point in a map is denoted by (x, y) co-ordinates. When we see kharagpur city on a scale
of India it will be seen as a point. Point can be used to denote various objects like origin,
city, end point, top of the mountain etc.

\item \textbf{Curve}\\
Curve is used to denote collection of points. This can be a straight line or curve. For
example, a road network can be represented with the help of line strings. Similarly, a
river can be denoted as a curve.

\item \textbf{Surface}\\
A surface is representation of an area or a polygon. When kharagpur is seen in the scale
of west Bengal it is seen as surface or polygon.

\item \textbf{GeomCollection}\\
Collection of basic building blocks defining a new type of geometry can be defined
with the help of GeomCollection.
\end{itemize}
NASA has a satellite called Earth Observation Satellite, which takes map images of earth and
sends it to observatory. It provides 3 terabytes(TB) of data on the daily basis as per the NASA.
This calculates to 90 TB data over a month and over a 1000 TB of data in a year. This data is
quite huge, it is unstructured in nature and it is un-indexed. Finding relevant information out
of this data is a non-trivial task. In the next section we see how the spatial data is accessed from
the internet.
\subsection{OGC Web Services}
Open Geospatial Consortium (OGC) is the worldwide standardization body for geospatial
standards. OGC provides a standardized way of accessing this geospatial data. OGC provides
three kinds of web services.
\begin{itemize}
\item \textbf{Web Map Service (WMS)}
Web Map service defines a way of accessing geospatial information across all geo servers in a
standard format as image. This image can be raster image or a vector image. Raster images are
of type jpg, png or bmp. Vector images contains svg format extension images. It also provides
a way to access metadata about the available information of the layers. This information can
contain type and no of layers. Some of the well-defined operations in this layer are
GetCapabilties, GetMap and DescribeLayer.
\item \textbf{Web Feature Service (WFS)}
WFS allows direct access to features contained in the map. WFS uses SOAP based interface.
For exchanging data between client and server WFS uses Geographical mark-up language
which is based on XML. Some well-defined operations in WFS are query or get feature, which
returns the feature stored on the server. We can add the feature in the repository by add feature.
We can delete feature by delete feature. Also we can update feature stored in the repository by
update feature command.
\item \textbf{Web Coverage Service (WCS)}
WCS offers multi-dimensional coverage of the geo spatial data. It provides spatio-temporal
context to the given geographical data. For example, it can show the flow of the river changing
over the span of years. Thus we can say that WCS provides richer coverage of spatial data than
WFS or WMS.
\end{itemize}

\subsection{Searching of spatial data}
There are previously known two popular and trivial approaches for searching spatial data.
\begin{enumerate}
\item \textbf{Catalogue approach}\\
Service provider registers their services to the registry. The registry contains various
type of registered services and their corresponding geo servers. But there are some
problems with this approach. First one is that in many cases the registry is not up to
date. Many times the latest services are not yet bound to any registry. One other problem
is related with incorrect classification of services. This is specially an issue because
user might be searching in the other part of the catalogue where it cannot find the
particular service even it is there. Last kind of problem can occur because not all kind
of providers registers all kind of services. Registry may be biased to some kind of
services.
\item \textbf{Utilization of popular search engine}\\
In an another approach we can also utilize popular search engines like Google, Yahoo,
Bing or DuckDuckGo to find spatial data. But the problem with this approach is that it
takes all data as general data and not spatial or other kind of unordinary data. Because
of this we lose many spatial operations and features. Another problem is the popular
search engines use some kind of ranking of pages which is not based on the quality of
the page(QoS). For example, google uses PageRank to determine the result webpages
but it is not dependent on the quality of the resultant webpage but merely a measure of
from how many other webpages the result webpage is addressed.
\end{enumerate}

% crawler part
\section{Crawler}
As we have seen, the above two methods fail to provide proper spatial context when searching
for the geospatial data. We need something that searches the whole web searches and preprocess the data available in the internet and give optimal suggestions about available spatial
service providers. A crawler is such kind of agent. It is a program that browse through the
world wide web in some systematic way and creates an index of the searched data. This index
can be later used for faster access for webpage or categorization of the webpage. Popular
examples of crawler programs are bingbot, googlebot and polybot.
\subsection{Types of web crawler}
There are mainly two types of web crawlers.
\begin{enumerate}
\item Universal crawler
\item Preferential crawler
\begin{itemize}
\item Focused crawler
\item Topical crawler
\end{itemize}
\end{enumerate}
Universal crawler is the crawler that treats all kind of webpages as same or given them same
weight. Other kind of crawler is a preferential crawler which prefers one kind of webpage over
another. One of the type of topical crawler is topical crawler. One type of topical crawler is
spatial crawler to which we will look into detail.
\subsection{Spatial Web Crawler: Objectives}
\begin{enumerate}
\item Build a spatial web crawler which crawlers through geo-servers which offers WFS
based OGC compliant services.
\item Build a domain specific vocabulary(ontology) for this features which can be helpful to
compare found features with wanted features.
\item Perform semantic matching of found features from crawled web-pages with given
ontology for filtering the correct features and storing them in the permanent repository.
\item Perform an evaluation of the given spatial web crawler using metrics and test URL seed
sets.
\end{enumerate}

\subsection{Architecture of the Spatial Web Crawler}
Our crawler contains three modules for crawling spatial features through the world wide web.
Some of the definition needed for understanding the working of spatial web crawler are:
\begin{itemize}
\item  \textbf{Seed set}: a set of test URLs to initialize the queue for crawling the web. The crawler
starts with the URLs given in the seed URL set.
\item \textbf{URLQueue}: A queue that contains set of URLs to be crawled. The crawler module takes
URLs one after the another from this queue.
\item \textbf{Frontiers}: a set of URLs from the URLQueue that are currently being crawled. This are
called frontiers because they reside between the known web and the unknown web.
\item \textbf{WMS resolver}: WMS resolver is a module that checks that if a server is WMS server
or not given the URL from the URLQueue.
\item \textbf{Parser}: parser is a module that downloads the webpage from the given URL and parse
the HTML webpage looking for pre-specified tags and names. After parsing it gets a
set of tags and its contents. In our implementation we parse the webpage and look for
the anchor tags within it and store all the hyperlinks from the crawled webpage.
\item \textbf{Ontology}: semantic dictionary containing all the features its type and relationship
hierarchy between features
\end{itemize}
Our crawler contains mainly three modules:
\begin{itemize}
\item \textbf{Extraction module}\\
Our algorithm starts with a set of seed URLs contained in the seed set. We initialize the
URLQueue with these seed URLs. The filter stage takes URL one after the other and checks
whether the given URL is already crawled or not. After filtering such URLs, they are sent to
the parser. The parser downloads the webpage from the given url and parses it for finding
hyperlinks contained in it. These hyperlinks again go to filter stage for finding whether the
given urls are already crawled or not. After filtering these urls are added to the end of
URLQueue. The filtered urls are also passed to the WFS module.
\item \textbf{WFS module}\\
Once we have a URL for the examination, we send a GetCapabilities() request to that server.
We do this by appending the request to the url.
$services?REQUEST=GetCapabilities\&version=1.1.0\&service =WFS$
The server replies for this request. If the reply contains WFS\_Capabilities tag, then it offers
WFS service. We parse the received response for finding the WFS\_Capabilities tag. If the given
server is WFS server then the given url is passed to analysis and indexing module.
\item \textbf{Analysis and Extraction module}\\
In this stage server response is parsed for the tag FeatureTypeList. FeatureTypeList
contains list of features. This features are stored under the tag FeatureType. FeatureType
tag contains set of keyword, title, name tags. Each of this tags are checked to see if it
contains any word from the ontology. For each of such tag found, DescribeFeatureType request
is appended to the url.

$�?service=WFS\&version=1.1.0\&request=DescribeFeatureType\&typename="+keyword$


Here the keyword is the name of the feature. Each of this retrieved feature is checked again
the ontology, if the feature is found in the ontology then it is added to permanent storage in
the repository.
\end{itemize}

\subsection{Advantages of Spatial Web Crawler}

\begin{itemize}
\item  Allows search in web pages that are not generally searchable from the normal search
engines. This is because of the spatial context awareness of the spatial web crawler.
\item It provides more up-to-date results from the results. Spatial crawler is more sensitive to
changes of spatial data on the web and automatically crawls through the changed
features with the help of automatic update module.
\item Provides improved accuracy in the search of spatial features and operations.
\item Provides extra features such as bounding box of spatially crawled data and other spatial
features.
\end{itemize}

\subsection{Challenges}
There are many challenges for practically implementing a crawler in the real world. Scale of
the web is so huge that the simple implementation of the theoretical crawler will be very poor.
One other problem is that many pages generate, add or delete on the dynamic bases. We need
to reflect this changes into the crawler. This is a crucial parameter for setting the update period
or refresh rate. Another problem arises because of heterogeneity in the available data on the
world wide web. How to register and store this data into the single repository depends on the
implementation of the algorithm.
\section{Evaluation of the system}
The performance of the system is measured taking various seed URLs, running the algorithm
against given sample data set in which correct results are already known. There are three kind
of measures used for measuring the performance of the spatial web crawler.
\begin{itemize}
\item \textbf{Precision}\\
Precision is found by dividing the no of geo servers found by the spatial web crawler by the
total no of geo servers found.

\begin{equation}
precision = \frac{no\ of\ relevant\ Geoservers}{total\ no\ of\ servers} * 100 \%
\end{equation}

\item \textbf{Recall}\\
Recall is found by taking a division of no of geo servers found by the spatial web crawler by
actual set of existing geo servers.
\begin{equation}
recall = \frac{no\ of\ relevant\ Geoservers}{total\ no\ of\ Geoservers}\ *\ 100 \%
\end{equation}

\item \textbf{F1 measure}\\
To normalize the values received by precision and recall parameters we generally use more
robust F1 measure.

\begin{equation}
F1 = 2\ *\ \frac{precision\ *\ recall}{precision\ +\ recall}
\end{equation}
\end{itemize}
Final evaluation is done by taking this measure for all the feature types and averaging it over
all the features.
\section{Conclusion}
Geo-spatial data is ever growing in today�s era. This type of data is hard to search from the
world wide web and index it for further processing. Our algorithm suggests a way for building
a WFS based Geo-crawler that efficiently crawls and indexes founded features into the
permanent repository. This repository can then be used in many applications like search
engines or data mining. This can help in many applications such as transportation \& navigation,
urban planning and emergency response planning.

% Now we need a bibliography:
\begin{thebibliography}{5}
% 1
\bibitem{latexcompanion} 
Patil, Sonal, Shrutilipi Bhattacharjee, and Soumya K. Ghosh. "A spatial web crawler for discovering geo-servers and semantic referencing with spatial features." Distributed Computing and Internet Technology. Springer International Publishing, 2014. 68-78.
% 2
\bibitem{latexcompanion}
Li, Wenwen, Chaowei Yang, and Chongjun Yang. "An active crawler for discovering
geospatial web services and their distribution pattern�a case study of OGC web map service." International Journal of Geographical Information Science 24.8 (2010): 1127-1147.
% 3
\bibitem{latexcompanion}
Jiang, Jun, Chong-jun Yang, and Ying-chao Ren. "A spatial information crawler for
opengis wfs." Sixth International Conference on Advanced Optical Materials and
Devices. International Society for Optics and Photonics, 2008.
%4
\bibitem{latexcompanion}
Marc Najork. �Web crawler architecture.� Microsoft Research.
% 5
\bibitem{latexcompanion}
Ahlers, Dirk, and Susanne Boll. "Location-based Web search." The Geospatial Web.
Springer London, 2009. 55-66.
% 6
\bibitem{latexcompanion}
Li, W., et al. "Semantic-based web service discovery and chaining for building an
Arctic spatial data infrastructure." Computers \& Geosciences 37.11 (2011): 1752-1762.
% 7
\bibitem{latexcompanion}
Jiang, Jun, Chong-jun Yang, and Ying-chao Ren. "A spatial information crawler for
opengis wfs." Sixth International Conference on Advanced Optical Materials and
Devices. International Society for Optics and Photonics, 2008.
% 8
\bibitem{latexcompanion}
Jiang, Jun, Chong-jun Yang, and Ying-chao Ren. "A spatial information crawler for
opengis wfs." Sixth International Conference on Advanced Optical Materials and
Devices. International Society for Optics and Photonics, 2008.
\end{thebibliography}

% Your document ends here!
\end{document}